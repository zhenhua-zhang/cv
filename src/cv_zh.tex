
%-----------------------------------------------------------------------------------------------------------------------------------------------%
%	The MIT License (MIT)
%
%	Copyright (c) 2021 Philip Empl
%
%	Permission is hereby granted, free of charge, to any person obtaining a copy
%	of this software and associated documentation files (the "Software"), to deal
%	in the Software without restriction, including without limitation the rights
%	to use, copy, modify, merge, publish, distribute, sublicense, and/or sell
%	copies of the Software, and to permit persons to whom the Software is
%	furnished to do so, subject to the following conditions:
%	
%	THE SOFTWARE IS PROVIDED "AS IS", WITHOUT WARRANTY OF ANY KIND, EXPRESS OR
%	IMPLIED, INCLUDING BUT NOT LIMITED TO THE WARRANTIES OF MERCHANTABILITY,
%	FITNESS FOR A PARTICULAR PURPOSE AND NONINFRINGEMENT. IN NO EVENT SHALL THE
%	AUTHORS OR COPYRIGHT HOLDERS BE LIABLE FOR ANY CLAIM, DAMAGES OR OTHER
%	LIABILITY, WHETHER IN AN ACTION OF CONTRACT, TORT OR OTHERWISE, ARISING FROM,
%	OUT OF OR IN CONNECTION WITH THE SOFTWARE OR THE USE OR OTHER DEALINGS IN
%	THE SOFTWARE.
%	
%
%-----------------------------------------------------------------------------------------------------------------------------------------------%


%============================================================================%
%
%	DOCUMENT DEFINITION
%
%============================================================================%

\documentclass[8pt,A4]{article}


%----------------------------------------------------------------------------------------
%	ENCODING
%----------------------------------------------------------------------------------------

% we use utf8 since we want to build from any machine
\usepackage[UTF8]{ctex}
%\usepackage[utf8]{inputenc}
\usepackage[USenglish]{isodate}
\usepackage{fancyhdr}
\usepackage[numbers]{natbib}

\setCJKmonofont{STFangsong}

%----------------------------------------------------------------------------------------
%	LOGIC
%----------------------------------------------------------------------------------------

% provides \isempty test
\usepackage{xstring, xifthen}
\usepackage{enumitem}
\usepackage[english]{babel}
\usepackage{blindtext}
\usepackage{pdfpages}
\usepackage{changepage}
\usepackage{setspace}
\setstretch{1.05}
%----------------------------------------------------------------------------------------
%	FONT BASICS
%----------------------------------------------------------------------------------------

% some tex-live fonts - choose your own

%\usepackage[defaultsans]{droidsans}
%\usepackage[default]{comfortaa}
%\usepackage{cmbright}
\usepackage[default]{raleway}
%\usepackage{fetamont}
%\usepackage[default]{gillius}
%\usepackage[light,math]{iwona}
%\usepackage[thin]{roboto} 

% set font default
\renewcommand*\familydefault{\sfdefault} 	
\usepackage[T1]{fontenc}

% more font size definitions
\usepackage{moresize}

%----------------------------------------------------------------------------------------
%	FONT AWESOME ICONS
%---------------------------------------------------------------------------------------- 

% include the fontawesome icon set
\usepackage{fontspec}
\usepackage{fontawesome5}

% use to vertically center content
% credits to: http://tex.stackexchange.com/questions/7219/how-to-vertically-center-two-images-next-to-each-other
\newcommand{\vcenteredinclude}[1]{\begingroup
\setbox0=\hbox{\includegraphics{#1}}%
\parbox{\wd0}{\box0}\endgroup}
\newcommand{\tab}[1]{\hspace{.2\textwidth}\rlap{#1}}
% use to vertically center content
% credits to: http://tex.stackexchange.com/questions/7219/how-to-vertically-center-two-images-next-to-each-other
\newcommand*{\vcenteredhbox}[1]{\begingroup
\setbox0=\hbox{#1}\parbox{\wd0}{\box0}\endgroup}

% icon shortcut
\newcommand{\icon}[3] { 							
	\makebox(#2, #2){\textcolor{maincol}{\csname fa#1\endcsname}}
}	


% icon with text shortcut
\newcommand{\icontext}[4]{ 						
	\vcenteredhbox{\icon{#1}{#2}{#3}}  \hspace{2pt}  \parbox{0.9\mpwidth}{\textcolor{#4}{#3}}
}

% icon with website url
\newcommand{\iconhref}[5]{ 						
    \vcenteredhbox{\icon{#1}{#2}{#5}}  \hspace{2pt} \href{#4}{\textcolor{#5}{#3}}
}

% icon with email link
\newcommand{\iconemail}[5]{ 						
    \vcenteredhbox{\icon{#1}{#2}{#5}}  \hspace{2pt} \href{mailto:#4}{\textcolor{#5}{#3}}
}

%----------------------------------------------------------------------------------------
%	PAGE LAYOUT  DEFINITIONS
%----------------------------------------------------------------------------------------

% page outer frames (debug-only)
% \usepackage{showframe}		

% we use paracol to display breakable two columns
\usepackage{paracol}
\usepackage{tikzpagenodes}
\usetikzlibrary{calc}
\usepackage{lmodern}
\usepackage{multicol}
\usepackage{lipsum}
\usepackage{atbegshi}
% define page styles using geometry
\usepackage[a4paper]{geometry}

% remove all possible margins
\geometry{top=1cm, bottom=1cm, left=0.75cm, right=1cm}

\usepackage{fancyhdr}
\pagestyle{empty}

% space between header and content
% \setlength{\headheight}{0pt}

% indentation is zero
\setlength{\parindent}{0mm}

%----------------------------------------------------------------------------------------
%	TABLE /ARRAY DEFINITIONS
%---------------------------------------------------------------------------------------- 

% extended aligning of tabular cells
\usepackage{array}

% custom column right-align with fixed width
% use like p{size} but via x{size}
\newcolumntype{x}[1]{%
>{\raggedleft\hspace{0pt}}p{#1}}%


%----------------------------------------------------------------------------------------
%	GRAPHICS DEFINITIONS
%---------------------------------------------------------------------------------------- 

%for header image
\usepackage{graphicx}

% use this for floating figures
% \usepackage{wrapfig}
% \usepackage{float}
% \floatstyle{boxed} 
% \restylefloat{figure}

%for drawing graphics		
\usepackage{tikz}			
\usepackage{ragged2e}	
\usetikzlibrary{shapes,backgrounds,mindmap,trees}

%----------------------------------------------------------------------------------------
%	Color DEFINITIONS
%---------------------------------------------------------------------------------------- 
\usepackage{transparent}
\usepackage{color}

% primary color
\definecolor{maincol}{RGB}{ 64,64,64}

% accent color, secondary
% \definecolor{accentcol}{RGB}{ 250, 150, 10 }

% dark color
\definecolor{darkcol}{RGB}{ 70, 70, 70 }

% light color
\definecolor{lightcol}{RGB}{245,245,245}

\definecolor{accentcol}{RGB}{59,77,97}



% Package for links, must be the last package used
\usepackage[hidelinks]{hyperref}

% returns minipage width minus two times \fboxsep
% to keep padding included in width calculations
% can also be used for other boxes / environments
\newcommand{\mpwidth}{\linewidth-\fboxsep-\fboxsep}
	


%============================================================================%
%
%	CV COMMANDS
%
%============================================================================%

%----------------------------------------------------------------------------------------
%	 CV LIST
%----------------------------------------------------------------------------------------

% renders a standard latex list but abstracts away the environment definition (begin/end)
\newcommand{\cvlist}[1] {
	\begin{itemize}{#1}\end{itemize}
}

%----------------------------------------------------------------------------------------
%	 CV TEXT
%----------------------------------------------------------------------------------------

% base class to wrap any text based stuff here. Renders like a paragraph.
% Allows complex commands to be passed, too.
% param 1: *any
\newcommand{\cvtext}[1] {
	\begin{tabular*}{1\mpwidth}{p{0.98\mpwidth}}
		\parbox{1\mpwidth}{#1}
	\end{tabular*}
}
\newcommand{\cvtextsmall}[1] {
	\begin{tabular*}{0.8\mpwidth}{p{0.8\mpwidth}}
		\parbox{0.8\mpwidth}{#1}
	\end{tabular*}
}
%----------------------------------------------------------------------------------------
%	CV SECTION
%----------------------------------------------------------------------------------------

% Renders a a CV section headline with a nice underline in main color.
% param 1: section title
\newcommand{\cvsection}[1] {
	\vspace{12pt}
	\cvtext{
    \textbf{\LARGE{\textcolor{darkcol}{#1}}}\\[-6pt]
    \textcolor{accentcol}{ \rule{0.2\textwidth}{1pt} } \\[-10pt]
	}
}

\newcommand{\cvsectionsmall}[1] {
	\vspace{8pt}
	\cvtext{
    \textbf{\Large{\textcolor{darkcol}{#1}}} \\[-6pt]
    \textcolor{accentcol}{ \rule{0.2\textwidth}{1pt} } \\[-10pt]
	}
}

\newcommand{\cvheadline}[1] {
	\vspace{14pt}
	\cvtext{
		\textbf{\Huge{\textcolor{accentcol}{#1}}}\\[-4pt]
	}
}

\newcommand{\cvsubheadline}[1] {
	\vspace{14pt}
	\cvtext{
		\textbf{\huge{\textcolor{darkcol}{#1}}}\\[-4pt]
	}
}
%----------------------------------------------------------------------------------------
%	META SKILL
%----------------------------------------------------------------------------------------

% Renders a progress-bar to indicate a certain skill in percent.
% param 1: name of the skill / tech / etc.
% param 2: level (for example in years)
% param 3: percent, values range from 0 to 1
\newcommand{\cvskill}[3] {
	\begin{tabular*}{1\mpwidth}{p{0.72\mpwidth}  r}
 		\textcolor{black}{\textbf{#1}} & \textcolor{maincol}{#2}\\
	\end{tabular*}%
	
	\hspace{1.5pt}
	\begin{tikzpicture}[scale=1,rounded corners=2pt,very thin]
		\fill [lightcol] (0,0) rectangle (1\mpwidth, 0.15);
		\fill [accentcol] (0,0) rectangle (#3\mpwidth, 0.15);
  	\end{tikzpicture}%
}


%----------------------------------------------------------------------------------------
%	 CV EVENT
%----------------------------------------------------------------------------------------

% Renders a table and a paragraph (cvtext) wrapped in a parbox (to ensure minimum content
% is glued together when a pagebreak appears).
% Additional Information can be passed in text or list form (or other environments).
% the work you did
% param 1: time-frame i.e. Sep 14 - Jan 15 etc.
% param 2: event name (job position etc.)
% param 3: Customer, Employer, Industry
% param 4: Short description
% param 5: work done (optional)
% param 6: technologies include (optional)
% param 7: achievements (optional)
\newcommand{\cvevent}[7] {
	% we wrap this part in a parbox, so title and description are not separated on a pagebreak
	% if you need more control on page breaks, remove the parbox
	\parbox{\mpwidth}{
		\begin{tabular*}{1\mpwidth}{p{0.78\mpwidth}  r}
	 		\textcolor{black}{\textbf{#2}} & \colorbox{accentcol}{\makebox[0.22\mpwidth]{\textcolor{white}{\textbf{#1}}}} \\
			\textcolor{maincol}{#3} & \\
		\end{tabular*}\\[1pt]
	
		\ifthenelse{\isempty{#4}}{}{
			\cvtext{#4}\\
		}
	}
	\vspace{5pt}
}


%----------------------------------------------------------------------------------------
%	 CV META EVENT
%----------------------------------------------------------------------------------------

% Renders a CV event on the sidebar
% param 1: title
% param 2: subtitle (optional)
% param 3: customer, employer, etc,. (optional)
% param 4: info text (optional)
\newcommand{\cvmetaevent}[4] {
	\textcolor{maincol} { \cvtext{\textbf{\begin{flushleft}#1\end{flushleft}}}}

	\ifthenelse{\isempty{#2}}{}{
	\textcolor{black} {\cvtext{\textbf{#2}} }
	}

	\ifthenelse{\isempty{#3}}{}{
		\cvtext{{ \textcolor{maincol} {#3} }}\\
	}

	\cvtext{#4}\\[14pt]
}

%---------------------------------------------------------------------------------------
%	QR CODE
%----------------------------------------------------------------------------------------

% Renders a qrcode image (centered, relative to the parentwidth)
% param 1: percent width, from 0 to 1
\newcommand{\cvqrcode}[1] {
	\begin{center}
		\includegraphics[width={#1}\mpwidth]{qrcode}
	\end{center}
}


% HEADER AND FOOOTER 
%====================================
\newcommand\Header[1]{%
\begin{tikzpicture}[remember picture,overlay]
\fill[accentcol]
  (current page.north west) -- (current page.north east) --
  ([yshift=50pt]current page.north east|-current page text area.north east) --
  ([yshift=50pt,xshift=-3cm]current page.north|-current page text area.north) --
  ([yshift=10pt,xshift=-5cm]current page.north|-current page text area.north) --
  ([yshift=10pt]current page.north west|-current page text area.north west) -- cycle;
\node[font=\sffamily\bfseries\color{white},anchor=west,
  xshift=0.7cm,yshift=-0.32cm] at (current page.north west)
  {\fontsize{12}{12}\selectfont {#1}};
\end{tikzpicture}%
}

\newcommand\Footer[1]{%
\begin{tikzpicture}[remember picture,overlay]
\fill[lightcol]
  (current page.south east) -- (current page.south west) --
  ([yshift=-80pt]current page.south east|-current page text area.south east) --
  ([yshift=-80pt,xshift=-6cm]current page.south|-current page text area.south) --
  ([xshift=-2.5cm,yshift=-10pt]current page.south|-current page text area.south) --	
  ([yshift=-10pt]current page.south east|-current page text area.south east) -- cycle;
\node[yshift=0.32cm,xshift=9cm] at (current page.south) {\fontsize{10}{10}\selectfont \textbf{\thepage}};
\end{tikzpicture}%
}


%============================================================================%
%	DOCUMENT CONTENT
%============================================================================%
\begin{document}

\columnratio{0.275}
\setlength{\columnsep}{2em}
\setlength{\columnseprule}{2pt}
\colseprulecolor{lightcol}


% LEBENSLAUF HIERE
\AtBeginShipoutFirst{\Header{张振华}\Footer{1}}
\AtBeginShipout{\AtBeginShipoutAddToBox{\Header{张振华}\Footer{2}}}


\begin{paracol}{2}
\begin{leftcolumn}
%---------------------------------------------------------------------------------------
%	META IMAGE
%----------------------------------------------------------------------------------------
\includegraphics[width=0.775\linewidth]{figures/zhenhua_zhang.jpg}	%trimming relative to image size

%---------------------------------------------------------------------------------------
%	META SKILLS
%----------------------------------------------------------------------------------------
\fcolorbox{white}{white}{\begin{minipage}[c][2cm][c]{1\mpwidth}
	\LARGE{\textbf{\textcolor{maincol}{张振华(博士)}}} \\[2pt]
	\normalsize{ \textcolor{maincol} {生物信息学与系统生物学} }
\end{minipage}}
% \icontext{CaretRight}{12}{UMC Groningen, NL}{black}\\[6pt]

%---------------------------------------------------------------------------------------
%	Basic information
%----------------------------------------------------------------------------------------
\cvsectionsmall{基本信息}

\cvtext{性别:男~~~民族:汉}

\cvtext{籍贯:山东济南}

\cvtext{出生日期:1990.02.01}

%---------------------------------------------------------------------------------------
%	Contact
%----------------------------------------------------------------------------------------
\cvsectionsmall{联系方式}

% \iconhref{Home}{12}{zhenhua-zhang.github.io}{https://zhenhua-zhang.github.io}{black}\\[5pt]
\icontext{MapMarker}{12}{Feodor-Lynen-Stra{\ss}e 7, \newline 30625, Hannover, DE}{black}\\[5pt]
\icontext{Phone}{12}{+86~189~1023~3783}{black}\\[5pt]
\iconhref{Envelope}{12}{zhenhua.zhang217@gmail.com}{zhenhua.zhang217@gmail.com}{black}\\[5pt]
\iconhref{Github}{12}{github.com/zhenhua-zhang}{https://www.github.com/zhenhua-zhang}{black}\\[5pt]


\cvsectionsmall{专业技能}

% Bioinformatics
<<<<<<< HEAD
\cvskill{Linux/Python/R/C}{10+年}{1} \\[-1pt]
\cvskill{生物信息学}{9+年}{.9} \\[-1pt]
\cvskill{基因组/转录组分析}{9+年}{0.9} \\[-1pt]
\cvskill{全基因组关联分析}{8+年}{0.7} \\[-1pt]
\cvskill{单细胞测序 / \newline RNA-seq/ATAC-seq}{6+年}{0.5} \\[-1pt]
\cvskill{机器/深度学习}{5+年}{0.4} \\[-1pt]
% \cvskill{深度学习}{5+年}{0.4} \\[-1pt]
=======
\cvskill{Linux/Python/R/C}{9+年}{1} \\[-1pt]
\cvskill{生物信息学}{8+年}{.9} \\[-1pt]
\cvskill{基因组/转录组分析}{8+年}{0.9} \\[-1pt]
\cvskill{全基因组关联分析}{7+年}{0.7} \\[-1pt]
\cvskill{单细胞测序 / \newline RNA-seq/ATAC-seq}{5+年}{0.5} \\[-1pt]
\cvskill{机器学习}{4+年}{0.4} \\[-1pt]
>>>>>>> fe1dfd13e01f6be5e7d704193743637b756ca64f

% Language
\cvskill{汉语普通话} {二甲} {1} \\[-2pt]
\cvskill{英语}{四六级}{0.80} \\[-2pt] %
\cvskill{英语(IELTS)} {6.5} {0.80} \\[-2pt] %

\newpage
%---------------------------------------------------------------------------------------
%	META SKILLS
%----------------------------------------------------------------------------------------
\fcolorbox{white}{white}{\begin{minipage}[c][2cm][c]{1\mpwidth}
	\LARGE{\textbf{\textcolor{maincol}{张振华(博士)}}} \\[2pt]
	\normalsize{ \textcolor{maincol} { 生物信息学与系统生物学 } }
\end{minipage}}


%---------------------------------------------------------------------------------------
%	Contact
%----------------------------------------------------------------------------------------
\cvsectionsmall{联系方式}

<<<<<<< HEAD
\iconhref{Home}{12}{zhenhua-zhang.github.io}{https://zhenhua-zhang.github.io}{black}\\[5pt]
=======
% \iconhref{Home}{12}{zhenhua-zhang.github.io}{https://zhenhua-zhang.github.io}{black}\\[5pt]
>>>>>>> fe1dfd13e01f6be5e7d704193743637b756ca64f
\icontext{MapMarker}{12}{Feodor-Lynen-Stra{\ss}e 7, \newline 30625, Hannover, DE}{black}\\[5pt]
\icontext{Phone}{12}{+86~189~1023~3783}{black}\\[5pt]
\iconhref{Envelope}{12}{zhenhua.zhang217@gmail.com}{zhenhua.zhang217@gmail.com}{black}\\[5pt]
\iconhref{Github}{12}{github.com/zhenhua-zhang}{https://www.github.com/zhenhua-zhang}{black}\\[5pt]


%\cvqrcode{0.3}

\end{leftcolumn}

\begin{rightcolumn}
%---------------------------------------------------------------------------------------
%	TITLE  HEADER
%----------------------------------------------------------------------------------------


%---------------------------------------------------------------------------------------
%	PROFILE
%----------------------------------------------------------------------------------------
% \cvsectionsmall{基本信息}
% \cvtext{姓名:张振华~~~~性别:男~~~~民族:汉~~~~出生日期:1990.02.01}


%---------------------------------------------------------------------------------------
%	PROJECTS
%----------------------------------------------------------------------------------------
\cvsectionsmall{科研项目}

\cvevent
{2019-2021}
{潜伏HIV病毒库数量性状位点鉴定}
{负责数量性状位点定位; 属于人类功能基因组计划的2000HIV研究(https://2000hiv.com)}
<<<<<<< HEAD
{\textbf{Zhang, Zhenhua}\textsuperscript{*}, Wim Trypsteen\textsuperscript{*} et al. \textit{IRF7 and RNH1 are modifying factors of HIV-1 reservoirs: a genome-wide association analysis.} BMC medicine 19, no. 1 (2021): 1-17. (JCR Q1, 中科院 Q1, 中科院top期刊, IF=11.15)}
=======
{\textbf{Zhang, Zhenhua}\textsuperscript{*}, Wim Trypsteen\textsuperscript{*} et al. \textit{IRF7 and RNH1 are modifying factors of HIV-1 reservoirs: a genome-wide association analysis.} BMC medicine 19, no. 1 (2021): 1-17. (JCR Q1, 中科院 Q1, 中科院top期刊)}
>>>>>>> fe1dfd13e01f6be5e7d704193743637b756ca64f
{}{}{}

\cvevent
{2021-2022}
{单细胞多组学整合鉴定新冠病毒易感性的(表观)遗传学基础}
{负责单细胞数据处理与整合, 等位染色质特异性开放位点鉴定}
<<<<<<< HEAD
{Zhang, Bowen\textsuperscript{*}, \textbf{Zhenhua Zhang}\textsuperscript{*}, Valerie A. C. M. Koeken\textsuperscript{*} et al. \textit{An altered and allele-specific open chromatin landscape reveals epigenetic and genetic regulators of innate immunity in COVID-19.} Cell Genomics (2022): 100232. (JCR Q1,中科院分区表Q1,中科院分区表top期刊,IF=11.10)}
{}{}{}

\cvevent
{2022-2025}
{单细胞水平基因表达数量性状位点(sceQTL)鉴定及其应用}
{数据分析,文章撰写等}
{Zhang, Zhenhua, et al. \textit{Unveiling genetic signatures of immune response in immune-related diseases through single-cell eQTL analysis across diverse conditions} Nature Communications (2025), 10.1038/s41467-025-61192-4,已接收. (JCR Q1,中科院分区Q1,TOP期刊,IF=14.70)}
{}{}{}

\cvevent
{2023-2024}
{单细胞水平免疫时钟阐明个体间免疫差异}
{Wenchao Li, Zhenhua Zhang, et al. \textit{Single-cell immune aging clocks reveal inter-individual heterogeneity during infection and vaccination} Nature Aging 5.4 (2025): 607. (JCR Q1,中科院分区Q1,TOP期刊,IF=14.70)}
{}{}{}{}

\cvevent
{2021-2024}
{HIV病毒携带者免疫细胞表面蛋白CCR5丰度数量性状位点鉴定}
{负责数量性状位点定位; 属于人类功能基因组计划的2000HIV研究(https://2000hiv.com); \textbf{已完成待发表}}
{}{}{}{}

\cvevent
=======
{Zhang, Bowen\textsuperscript{*}, \textbf{Zhenhua Zhang}\textsuperscript{*}, Valerie A. C. M. Koeken\textsuperscript{*} et al. \textit{An altered and allele-specific open chromatin landscape reveals epigenetic and genetic regulators of innate immunity in COVID-19.} Cell Genomics (2022): 100232. (JCR Q1)}
{}{}{}

\cvevent
{2021-2023}
{HIV病毒携带者免疫细胞表面蛋白CCR5丰度数量性状位点鉴定}
{负责数量性状位点定位; 属于人类功能基因组计划的2000HIV研究(https://2000hiv.com); \textbf{已完成待发表}}
{}{}{}{}

\cvevent
{2022-至今}
{单细胞水平基因表达数量性状位点(sceQTL)鉴定及其应用}
{负责基因表达数量性状位点定位, 单细胞多组学整合(scRNA-seq/scATAC-seq); \textbf{已完成待发表}}
{}{}{}{}

\cvevent
>>>>>>> fe1dfd13e01f6be5e7d704193743637b756ca64f
{2023-至今}
{进行性脑白质病变(PML)单细胞水平的机理研究}
{负责单细胞水平差异基因鉴定, 单细胞多组学整合(scRNA-seq, scATAC-seq, scCITE-seq, sc-multiome); \textit{进行中}}
{}{}{}{}

\cvevent
{2023-至今}
{HIV病毒携带者的全基因组测序(WGS)}
{负责全基因组测序, 遗传变异鉴定(SNP, INDEL, SV, CNV), CNV与精英控制者(elite-controllers)的关系研究; 属于人类功能基因组计划的2000HIV研究(https://2000hiv.com); \textit{已完成}}
{}{}{}{}

\cvevent
{其它}
{枯草芽孢杆菌的益生菌潜力; 稻瘟菌中硫醇蛋白的氧化还原蛋白质组学; 等位基因特异性表达的预测}
{负责微生物基因组拼装, 遗传变异鉴定, 进化分析; 蛋白质组数据分析, 功能蛋白鉴定; 等位基因特异性表达鉴定及预测}
{Yi, Yanglei\textsuperscript{*}, \textbf{Zhenhua Zhang}\textsuperscript{*} et al. "Probiotic potential of Bacillus velezensis JW: antimicrobial activity against fish pathogenic bacteria and immune enhancement effects on Carassius auratus." Fish \& shellfish immunology 78 (2018): 322-330. \\
  Zhang, Xinrong\textsuperscript{*}, \textbf{Zhenhua Zhang}\textsuperscript{*} et al. \textit{The redox proteome of thiol proteins in the rice blast fungus Magnaporthe oryzae.} Frontiers in Microbiology 12 (2021): 278. \\
  \textbf{Zhang, Zhenhua} et al. \textit{Feasibility of predicting allele specific expression from DNA sequencing using machine learning.} Sci Rep 11, no. 1 (2021): 1-11. \\
}
{}{}{}



%---------------------------------------------------------------------------------------
%	EDUCATION & WORK EXPERIENCE
%----------------------------------------------------------------------------------------
\cvsectionsmall{教育背景及工作经历}

\cvevent
<<<<<<< HEAD
{2024.05-至今}
{副研究员 | 中国科学院广州生物医药与健康研究员}
{细胞谱系研究中心,广东省广州市}
{}{}{}{}

\cvevent
=======
>>>>>>> fe1dfd13e01f6be5e7d704193743637b756ca64f
{2022.05-2024.04}
{博士后 | Singh-Chhatwal-Postdoctoral Fellowship Program}
{精准感染医学中心(CiiM, \textit{Prof.} Y. Li), 德国亥姆霍兹传染研究中心 (HZI)}
{}{}{}{}

\cvevent
{2017.10-2022.04}
{博士 | 生物信息学与系统生物学}
{遗传学系(Dep. of Genetics, \textit{Prof.} M. Swertz \& \textit{Prof.} Y. Li),荷兰格罗宁根大学 (UG)}
{}{}{}{}

\cvevent
{2015.09-2017.06}
{硕士 | 生物信息学与植物保护学}
{植物保护学院(\textit{Prof.} 彭友良), 中国农业大学}
{}{}{}{}

\cvevent
{2014.09-2015.06}
{支教 | 西部计划}
{平旺中学, 防城港市, 广西壮族自治区}
{}{}{}{}

\cvevent
{2010.09-2014.06}
{学士 | 植物保护学}
{农学与生物技术学院, 中国农业大学}
{}{}{}{}


% \newpage
%---------------------------------------------------------------------------------------
%	Publications and thesis
%----------------------------------------------------------------------------------------
% \cvsectionsmall{学术论文}
% 
% \cvtextsmall{* \textit{共同第一作者。}}
% \begin{itemize}[leftmargin=2em,itemsep=-2pt]
  % \item \textbf{Zhang, Zhenhua}\textsuperscript{*}, Wim Trypsteen\textsuperscript{*}, Marc Blaauw, Xiaojing Chu, Sofie Rutsaert, Linos Vandekerckhove, Wouter van der Heijden et al. \textit{IRF7 and RNH1 are modifying factors of HIV-1 reservoirs: a genome-wide association analysis.} BMC medicine 19, no. 1 (2021): 1-17.
  % \item Zhang, Bowen\textsuperscript{*}, \textbf{Zhenhua Zhang}\textsuperscript{*}, Valerie A. C. M. Koeken\textsuperscript{*}, Saumya Kumar, Michelle Aillaud, Hsin-Chieh Tsay, Zhaoli Liu, et al. \textit{An altered and allele-specific open chromatin landscape reveals epigenetic and genetic regulators of innate immunity in COVID-19.} Cell Genomics (2022): 100232.
  % \item Yi, Yanglei\textsuperscript{*}, \textbf{Zhenhua Zhang}\textsuperscript{*}, Fan Zhao, Huan Liu, Lijun Yu, Jiwei Zha, and Gaoxue Wang. "Probiotic potential of Bacillus velezensis JW: antimicrobial activity against fish pathogenic bacteria and immune enhancement effects on Carassius auratus." Fish \& shellfish immunology 78 (2018): 322-330.
  % \item Zhang, Xinrong\textsuperscript{*}, \textbf{Zhenhua Zhang}\textsuperscript{*}, and Xiao-Lin Chen. \textit{The redox proteome of thiol proteins in the rice blast fungus Magnaporthe oryzae.} Frontiers in Microbiology 12 (2021): 278.
  % \item \textbf{Zhang, Zhenhua}, Freerk van Dijk, Niek de Klein, Mariëlle E. van Gijn, Lude H. Franke, Richard J. Sinke, Morris A. Swertz, and K. Joeri van der Velde. \textit{Feasibility of predicting allele specific expression from DNA sequencing using machine learning.} Sci Rep 11, no. 1 (2021): 1-11.
  % \item dos Santos, Jéssica C.\textsuperscript{*}, \textbf{Zhenhua Zhang}\textsuperscript{*}, Louise E. van Eekeren, Ezio T. Fok, Nadira Vadaq, Lisa van de Wijer, et al. \textit{Host genetic variants regulates CCR5 expression on immune cells: a study in people living with HIV and healthy controls.} \textit{BioRxiv} (2022)
  % \item Li, Yu, Jing Zhao, Spyridon Achinas, \textbf{Zhenhua Zhang}, Janneke Krooneman, and Gert Jan Willem Euverink. \textit{The biomethanation of cow manure in a continuous anaerobic digester can be boosted via a bioaugmentation culture containing Bathyarchaeota.} Science of the Total Environment 745 (2020): 141042.
  % \item Li, Yu, Jing Zhao, and \textbf{Zhenhua Zhang}. \textit{Implementing metatranscriptomics to unveil the mechanism of bioaugmentation adopted in a continuous anaerobic process treating cow manure.} Bioresource Technology 330 (2021): 124962.
  % \item Li, Yu, and \textbf{Zhenhua Zhang}. \textit{Recognize the benefit of continuous anaerobic co-digestion of cow manure and sheep manure from the perspective of metabolic pathways as revealed by metatranscriptomics.} Bioresource Technology Reports 17 (2022): 100910.
  % \item Aznaourova, Marina, Nils Schmerer, Harshavardhan Janga, \textbf{Zhenhua Zhang}, Kim Pauck, Judith Hoppe, Sarah M. Volkers et al. \textit{Single cell RNA-seq uncovers the nuclear decoy lincRNA PIRAT as a regulator of systemic monocyte immunity during COVID-19.} Proceedings of the National Academy of Sciences 119, no. 36 (2022): e2120680119 (2022).
  % \item Yang, Fan, Hans Meerman, \textbf{Zhenhua Zhang}, Jianrong Jiang, André Faaij. "Integral techno-economic comparison and GHG balances of different production routes of aromatics from biomass with CCS." Journal of Cleaner Production no. 372 (2022): 133727
  % \item van Eekeren, Louise E., Vasiliki Matzaraki, \textbf{Zhenhua Zhang}, Lisa Van De Wijer, Marc Blaauw, Marien I De Jonge, Linos Vandekerckhove et al. \textit{People with HIV using long-term cART have higher percentages of circulating CCR5+ CD8+ T cells and lower percentages of CCR5+ regulatory T cells compared to healthy controls - a cross-sectional study.} Sci Rep 12, 11425 (2022)
% \end{itemize}


% \newpage
%---------------------------------------------------------------------------------------
%	CONFERENCE, AWARDS
%----------------------------------------------------------------------------------------
\cvsectionsmall{其它}

\cvevent
{2018-2025}
{学术会议}
{
  全国生物信息学与系统生物学大会,海口,海南省,中国 \newline
  RNA Biology (Cold Spring Harbor Asia),Suzhou,China \newline
  ASHG Conference 2022, Los Angeles, US \newline
  ESHG Conference 2021, virtual \newline
  The Complex Life of RNA 2020, virtual \newline
  ESHG Conference 2020, virtual \newline
  BioSB Conference 2019, Lunteren, NL \newline
  KVMM Spring Meeting 2018, Papendal, NL
}
{}{}{}{}

\cvevent
{2011-2024}
{获奖}
{
  创新型人才国际合作培养项目, 2017, 中国国家留学基金委,北京 \newline
  正大奖学金, 2016, 中国农业大学,海淀,北京 \newline
  学习优秀二等奖学金, 2015, 中国农业大学,海淀,北京 \newline
  优秀教师, 2015,平旺中学, 防城港市,广西 \newline
  社会活动一等奖学金, 2013, 中国农业大学,海淀,北京 \newline
  优秀学生干部, 2013, 中国农业大学,海淀,北京 \newline
  学习优秀三等奖学金, 2011, 中国农业大学,海淀,北京
}
{}{}{}{}


% \newpage
% %---------------------------------------------------------------------------------------
% %	Course
% %----------------------------------------------------------------------------------------
% \vspace{2pt}
% \cvsectionsmall{课程}
% 
% \cvevent
% {Mar/2021 - Apr/2021}
% {Appliced Longitudinal Data Analysis: modeling change over time - ONLINE}
% {Graduate School of Medical Science}
% {University of Groningen}
% {}{}{}
% 
% \cvevent
% {Dec/2020}
% {Epidemiology and Appplied Statistics - ONLINE}
% {University Medical Center Groningen}
% {}{}{}{}
% 
% \cvevent
% {Jan/2019}
% {Medical Statistics}
% {University Medical Center Groningen}
% {}{}{}{}
% 
% \cvevent
% {Dec/2018}
% {Advances in Genetic Epidemiological Research and Data Analysis}
% {University Medical Center Groningen}
% {}{}{}{}


% \newpage
% %---------------------------------------------------------------------------------------
% %	Workshop/Seminar
% %----------------------------------------------------------------------------------------
% \vspace{5pt}
% \cvsectionsmall{Workshop/Seminar}
% 
% \cvevent
% {17/Nov/2021}
% {Workshop Peter Visscher}
% {University Medical Center Groningen, Groningen, NL}
% {Online}
% {Peter Visscher}{}{}
% 
% \cvevent
% {15/Nov/2021}
% {Seminar Peter Visscher}
% {University Medical Center Groningen, Groningen, NL}
% {Online}
% {Peter Visscher}{}{}
% 
% \cvevent
% {29/Oct/2020}
% {Introduction to Bayesian statistics}
% {University Medical Center Groningen, Groningen, NL}
% {Online workshop}
% {Dr Jorge Tendeiro}
% {}{}

% Groningen, \today
% \hspace*{1cm} \hrulefill
% 
% \hspace*{2cm}
% \phantom{Groningen, \today } Zhenhua Zhang

\end{rightcolumn}
\end{paracol}


\end{document}
